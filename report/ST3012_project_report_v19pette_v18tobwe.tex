%%%%% Single page layout:
%%%%% ----------------------------------------------------
\documentclass[12pt,a4paper,paper=a4,oneside,titlepage,pdftex]{scrartcl} 

%%% Additional useful packages
%%% ----------------------------------------------------------------
\usepackage{amsmath,amssymb,amsfonts}
\usepackage{algorithmic}
\usepackage{graphicx}
\usepackage{multirow}
\usepackage{array}
\usepackage{listings}                
\usepackage{subcaption}
\usepackage{float}
\usepackage[utf8]{inputenc}
\usepackage{booktabs}
\usepackage{pdfpages}
\usepackage{hyperref}
\usepackage{url}
\usepackage{xcolor} %red, green, blue, yellow, cyan, magenta, black, white
\usepackage{natbib}
%\usepackage{harvard}

%table caption on top
\usepackage{float}
\floatstyle{plaintop}
\restylefloat{table}

%custom command to rotate text
\newcommand*\rot{\rotatebox{90}}

\lstset{ 
  language=R,                     % the language of the code
  basicstyle=\tiny\ttfamily, % the size of the fonts that are used for the code
  numbers=left,                   % where to put the line-numbers
  numberstyle=\tiny\color{black},  % the style that is used for the line-numbers
  stepnumber=1,                   % the step between two line-numbers. If it is 1, each line
                                  % will be numbered
  numbersep=5pt,                  % how far the line-numbers are from the code
  backgroundcolor=\color{white},  % choose the background color. You must add \usepackage{color}
  showspaces=false,               % show spaces adding particular underscores
  showstringspaces=false,         % underline spaces within strings
  showtabs=false,                 % show tabs within strings adding particular underscores
  frame=single,                   % adds a frame around the code
  rulecolor=\color{black},        % if not set, the frame-color may be changed on line-breaks within not-black text (e.g. commens (green here))
  tabsize=2,                      % sets default tabsize to 2 spaces
  captionpos=b,                   % sets the caption-position to bottom
  breaklines=true,                % sets automatic line breaking
  breakatwhitespace=false,        % sets if automatic breaks should only happen at whitespace
  keywordstyle=\color{black},      % keyword style
  commentstyle=\color{black},   % comment style
  stringstyle=\color{black}      % string literal style
} 

\begin{document}
	
\pagenumbering{arabic}

\title{Project Report: Users' satisfaction about Dalarna University's Homepage}
\subtitle{ST3012 Data Collection}
\author{
	\bfseries\Large Authors: Péter Tempfli, Tobias Weiß\\
	\{v19pette, v18tobwe\}@du.se
	\\ \\
	\includegraphics[]{figures/du-logo.jpg}\\
}

\maketitle
\tableofcontents
\newpage

\section{Introduction}
The website of an organization is its figurehead. Not long time ago, Dalarna University decided to update the website layout to cutting-edge web technologies. To set up a website is easy but who guarantees that the users are satisfied with it?

\subsection{Research questions}
Users' satisfaction is a very broad topic and the scope has to be narrowed down carefully. A survey about general satisfaction would be doomed to fail. It either would be too big and comprehensive or too general and vague. Both would lead to missing values and imprecise information. Therefore, this project focuses on aspects of the "search personal" page. In oder to answer these questions a survey among the website users is conducted. Following main questions are raised:

\begin{enumerate}
	\item \textbf{User friendliness:} When "find personal" is accessed, the navigation path is intuitive, quick and the front-end appears up-to-date?
	\item \textbf{Efficiency:} If "find personal" page is visited, then the required data is found quickly?
	\item \textbf{Filtering facility:} If a filter is used (added or deleted), is it clear which relation/hierarchy between filters prevails?
\end{enumerate}

\subsection{Goals}
In order to get deeper knowledge of the user needs a series of steps have to be executed:
\begin{itemize}
	\item Define methodology, including sampling size, experiment design, survey tool
	\item Design the survey form, including survey questions guideline
	\item Assure data quality via DESAP checklist
	\item Analyze data with the prepared methodology
	\item Present the findings within report and presentation
\end{itemize}

\subsection{Final aim}
With this project we help to understand how well user needs are suited by the current "find personal" system. The project shows where the weaknesses and strengths are. It can be used to determine if changes are necessary or not.

\section{Methodology}

\subsection{Designing the experiment}
In order to have a correct, measurable outcome, the environment of the experiment should be designed using well known, scientifically reliable techniques. We have to ensure the following principles:
\begin{enumerate}
\item \textbf{Randomization}.  We need to have objective effect, yet also need to give room for statistical inference
\item\textbf{Replicable experiment} Our experiment should be replicable; so we can study different outcomes from the \textit{same experiment}.
\item \textbf{Controllable} A designed experiment makes sure that the effect are caused by factors, which are known by the researchers.
\end{enumerate}

We are testing one environment, but we have more than one sub-sections in our population group. We can argue that we have \textit{at least} one dimension of sub-group in this population (status: student, teacher, researcher...); and it is also easy to see, why we could create more than one dimensions of the sub-populations (age, sex, etc.).

The other dimension is the behavior which site visitors conduct on the site. Not every visitor behaves the same way, thus some visitors experience aspect of the site, which others might not. The list of possible behaviors is probably endless, so we need to focus on very broad, but still distinctive usage patterns, such as "Simple person search, user found", "Simple person search, filter used" and "Simple person search, combination of filters used".

We can see that we have \textit{at least }2 sources of variability : user group and user behavior. It is impossible to test every combination of these blocks, but if we still want to know, which block is responsible for some effect in the final outcome. For our case, if we stick with the 2-dimensional population model, the Latin Square Design is a good candidate.

The LSD design (see formula \ref{eq:LSD-design}) orders outcome data into tables, and from these matrices builds a model, which is used for a linear regression:

\begin{equation}
y_{i,j,c} = \mu + \tau j + \beta i + \gamma c + \epsilon_{i,j,c}
\label{eq:LSD-design}
\end{equation}

If we only calculate the mean of the whole survey, we can't tell anything specific about sub-groups. However, once we have the regression ran, we have all the coefficients for every block; so in our case, we know, how much a sub-group likes more, or likes less a given feature. The linear model also tells the significance levels of these coefficients.

\subsection{Population}
The population are all human beings who have visited the Dalarna University website in the past and who will visit it in the future. One can estimate their number by analyzing server logs but still has to narrow down the time (visitors during the last month) and make certain assumptions.

\subsection{Sampling size}
As no previous data is available, one has to assume the most conservative P value of $ P = 0.5 $. Furthermore, the standard values of 10\% for error level and 95\% for the confidence interval are assumed. Therefore, the sampling size is calculated as shown in equation \ref{eq:sampling-size}.

\begin{equation}
n = Z^2_\alpha \frac{P\dot(1-P)}{(\hat{p}-P)^2} = 1.96^2\frac{0.5\dot{(1-0.5)}}{0.1^2}=96 \text{ subjects}
\label{eq:sampling-size}
\end{equation}

\subsection{Quantification and scale}

A qualitative survey \citep{sofaer1999qualitative} with quantified questions is suggested as functionality remains a subjective impression. For quantification a five-point Likert scale \citep{likert1932technique} is used. The scale is not directly visible to the subject as questions can be answered with sentences like "I fully agree", "I agree", etc. It is quantified in a bipolar way as shown in figure \ref{tab:response-scale}.  There is a discussion \citep{cummins2000we} whether five-point Likert scales are a sound method or not. This research sticks to it as an seven-point Likert scale \citep{chimi2009likert} or even a tailor-made scale seem unhandy and not easily adaptable in the system, which is used for data collection.

\begin{table}[h!]
\begin{tabular}{|c|| c|c|c|c|c|}
\hline
\textbf{Semantic text} & I fully agree & I agree & I mainly agree & I partly disagree & I disagree \\ \hline
\textbf{Quantification} & 2 & 1 & 0 & -1 & -2 \\ \hline
\end{tabular}
	\caption{Response scale}
	\label{tab:response-scale}
\end{table}


\subsection{Survey tool}
A Google docs online form is chosen to conduct the research. It seems to be used in other educational context \citep{gehringer2010daily} but is not suggested in research beyond a Master thesis level. Still, the use of Google forms remains questionable. There are many arguments pro and contra using it. On one hand using Google forms is cheap, easy and fast. The data might even be more secure as on local servers as Google probably has good security and backup strategies. On the other hand one gives the data away to Google and can not be fully aware anymore what happens to it. Dalarna University also offers a survey tool which should be used instead in a real context.

\subsection{Data collection plan}
As the survey is not carried out on the full sample size but just on a few show cases, we send an E-Mail or WhatsApp message with link to the Google forms document and explanation to some of our classmates.

In reality, an implementation of sequential sampling as described by Fan et al. \citep{fan1962development} is suggested directly on the website. If users search persons, a binomial approach can be applied. If the user is selected, an overlay with an invitation to the survey can be displayed. Still, it remains questionable if the structure of respondents is representative and delivers an unbiased view as humans tend only to report unwanted results \citep{bergstrand1983bias}. Maybe incentives to answer the questionnaire have to be introduced.

\subsection{Statistical indicators}
The analysis will heavily relate on cross tables. Besides means and medians, $\chi^2$ coefficients \citep{doi:10.1080/14786440009463897} (as shown in formula \ref{eq:chi-squared}) and contingency coefficients \citep{pearson1930theory} (as shown in formula \ref{eq:contingency-coefficient}) are calculated in order to determine the level of independence among the variables. These measures are important to see whether the questions are answered in a different way among the user groups (pupils, students, researchers, etc). Technically, the calculations will be performed in R using the \verb|assocstats| function from the package \verb|vcd| \citep{vcd2006}.

\begin{equation}
\chi^2 = \sum_{r=1}^{k}\sum_{s=1}^{t}\frac{(m_{r,s}-n_{r,s})^2}{m_{r,s}}
\label{eq:chi-squared}
\end{equation}

\begin{equation}
K = \sqrt{\frac{\chi^2}{n+\chi^2}}
\label{eq:contingency-coefficient}
\end{equation}@Misc{,
  Owner                    = {homaar},
  Timestamp                = {2019.05.14}
}


A pairwise correlation coefficient \citep{lee1988thirteen} (as shown in equation \ref{eq:correlation}) is computed to show if some of the questions can be abandoned in the future.

\begin{equation}
r_{x,y} = \frac{s_{x,y}}{\sqrt{s_{x^2} \cdot s_{y^2}}}
\label{eq:correlation}
\end{equation}

\subsection{Data quality assurance}
The DESAP checklist is used to keep track of the Project's data quality. "DESAP  is  the  generic  checklist  for  a  systematic  quality  assessment  of  surveys  in  the  European Statistical  System  (ESS).  It  has  been  designed  as  a  tool  for  survey  managers  and  should  support them  in  assessing  the  quality  of  their  statistics  and  considering  improvement  measures.  During  its development, the checklist has been tested in a pilot study covering a large variety of survey areas. It is fully compliant with the ESS quality criteria and comprises the main aspects relevant to the quality of  statistical  data \citep{desap2019}".

\subsection{Ethical considerations} 
While the main goal of a research is to answer questions and provide meaningful results, it is also crucial that we conduct the research in an ethical manner. 

As American Statistical Association’s Ethical Guidelines (\citep{asa}) states, an ethical research should be transparent, reproducible and valid. The researchers should also think ethically about every stakeholder of the given project in order to avoid or minimize harm; and not to pursuit unethical ends. 

The Guidelines of the American Statistical Association are mostly similar to the guidelines of the RESPECT code (\citep{respect}); the main ideas are the following:

\begin{itemize}
	\item \textbf{Consideration about the data}. The data should be objective and valid. The researcher should know the implicit biases and limitations of the data.
	\item \textbf{The research subject}. The researcher should consider the rights and the interests of the involved research subjects.
	\item \textbf{The team colleagues}. The research is a team-work, so the researcher should consider the rights and interests of the team members.
	\item \textbf{Other statisticians}. A researcher should share data and methods if possible; should consider the interest of the scientific community.
	\item \textbf{Employers, customers or other commercial stakeholders.} The researcher should work in a moral way, also in a commercial environment, and should strive to professionalism.
\end{itemize}

In terms of our project, the we should consider the following:
\begin{itemize}
	\item We design an experiment where the data is not biased; so we need to use proven experiment design techniques.
	\item We do not cause any harm to the university website, so we should think, if the extra traffic by our experiment slow down the servers.
	\item We respect the right and interests of the people involved; so need to make sure, that all of their personal data and behavior during the research is cared carefully.
	\item As the key question in the survey is how to improve it's' website, we need to try to come up with meaningful, valid, and valuable answers.
	\item During the research we need to make sure that all of the university employees, as well as the team members are cared morally.
	\item The team members should act professionally, so they won't hurt the image and the interests of the scientific community.
\end{itemize}

\subsection{Legal considerations}

As we research European Union subjects, and we store their personal data, we need to act accordingly to the The EU General Data Protection Regulation (GDPR) \citep{voigt2017eu}. The researchers in this case are the Data Processors; because they are handling and storing the subjects's personal data, which is in this case is 1, Name 2, Email 3, IP address. According to the law this is personal data, because the subjects can be directly or indirectly identified using these data points. It is important to understand that pseudonymised can also count as personal data; and the process itself of data-transformation is a personal data handling act.

The GDPR demands from the researchers that the handle of personal data should be lawful, fair and transparent. These guidelines can be found at universities and at other research groups.

In order to deal with the personal data, first of all, the subject should give a consent. The consent should be:
\begin{itemize}
	\item Freely given
	\item Specific
	\item Informative
	\item Unambigous
	\item An affirmative action	
\end{itemize}

There can be other bases for personal data processing (Contract, Legal Obligation, Vital interest, Public task, Legitimate interest), but we are not dealing with these here in detail.

In our case, in the survey, on the first question, we need to describe the goals of our research; the data which we are collecting about the individuals; and as for a consent with a check box which states 'Yes, I agree'. Once the research subject agreed, we can follow with the following questions.

\section{Analysis}

\subsection{Pilot}
Three students were asked to fill out the form in order to check whether the structure and questions are clear and to avoid all different forms of pitfalls. 

We discovered that the e-mail address field is not verified and can contain wrong information. Furthermore, it is not checked if the person already filled a form. This could be solved via a limitation to one response (see figure \ref{fig:response-limit}). The problem is that subjects need a Google account to fill the form if this option is selected. In reality, the suggested pop-up on the website should have a token which becomes invalid after a certain period or when the form was filled once. The email verification could be accomplished via a verification  e-mail containing a link which might be mandatory to open before filling the answers.

Regarding the subject class it turned out that some subjects chose the wrong category. Therefore, the order of possible answers will be rearranged.

\begin{figure}[h!]
    \centering
    \includegraphics[width=0.8\textwidth]{figures/response-limitation.png}
    \caption{Response limitation}
    \label{fig:response-limit}
\end{figure}

\subsection{Analysis plan}
All questions follow the same scheme. A dummy cross table for all single questions is represented by table \ref{tab:dummy-crosstable}. All results will be displayed using slightly modified versions of this table. The overall satisfaction is calculated using a similar crosstable (as shown in table \ref{tab:dummy-crosstable-summary}). Google forms offers an easy CSV export function. The data is downloaded and imported into RStudio. Subsequently statistical indicators as listed in the methodology are calculated.

\begin{table}[h!]
	\centering
    \begin{tabular}{ | r || c | c | c | c | c || r |}
      \hline
        & \rot{I fully agree} & \rot{I agree}  & \rot{I mainly agree}  & \rot{I partly disagree}  & \rot{I disagree} & $\Sigma$ \\ \hline \hline
      pupil & 0 & 0 & 0 & 0 & 0 & 0 \\ \hline
      student from other institution & 0 & 0 & 0 & 0 & 0 & 0 \\ \hline
      researcher from other institution & 0 & 0 & 0 & 0 & 0 & 0 \\ \hline
      student @DU & 0 & 0 & 0 & 0 & 0 & 0 \\ \hline
      researcher/employee @DU & 0 & 0 & 0 & 0 & 0 & 0 \\ \hline
      other & 0 & 0 & 0 & 0 & 0 & 0 \\ \hline \hline
      $\Sigma$ & 0 & 0 & 0 & 0 & 0 & 0 \\ \hline
    \end{tabular}
  \caption{Dummy cross table for all questions}
  \label{tab:dummy-crosstable}
\end{table}


\begin{table}[h!]
	\centering
    \begin{tabular}{ | r || c | c | c | c | c || r |}
      \hline
        & \rot{I fully agree} & \rot{I agree}  & \rot{I mainly agree}  & \rot{I partly disagree}  & \rot{I disagree} & $\Sigma$ \\ \hline \hline
      Question 1 & 0 & 0 & 0 & 0 & 0 & 0 \\ \hline
      \dots & 0 & 0 & 0 & 0 & 0 & 0 \\ \hline
      Question 5 & 0 & 0 & 0 & 0 & 0 & 0 \\ \hline
      $\Sigma$ & 0 & 0 & 0 & 0 & 0 & 0 \\ \hline \hline
    \end{tabular}
  \caption{Dummy cross table for overall satisfaction}
  \label{tab:dummy-crosstable-summary}
\end{table}

First, the collected data has to be transformed into a numeric form. The code as shown in listing \ref{lst:r-import} (see appendix) shows the fist step.

Then we are calculating the means and the medians for all questions as well as for every group. This will show the biggest issues for the several user groups.

We calculate the $\chi^2$ and contingency coefficient for all questions as well as the summarized satisfaction in order to discover the level of independence.

Next, we calculate the correlations between the different answers for the different questions (e.g. is there a correlation between Loading time and immediate finding of information). Furthermore, correlations are calculated for answers and groups (e.g. pupils might have answered differently than researchers).

\section{Conclusion}

After Dalarna University's website changed, our research tried to generate meaningful information about user satisfaction of the "search personal" function. We focused on  three aspects: \textit{user friendliness}, \textit{efficiency} and the \textit{filtering facilities}. The final aim of the research is to provide the website administrators and developers feedback, which show the critical points, as well as the strengths of the system.

The data collection and the research methodology use proved and scientific procedures, so the test about user satisfaction can be repeated. This can be needed, if some changes would been made in the system. By repeating the research using the same methodology, the effectiveness of the changes could also be measured.

We created conclusions from 2 perspectives:
\begin{itemize}
	\item \textbf{User groups:} Different type of users have different needs, and they behave differently. Our research shows, which functionalities are the most important / satisfactory for different site visitors.
	\item \textbf{Site functionalities:} We also tested the different functionalities (described in the questions) independently; so we see their overall correctness. Also, we examined how these functions work together from the user satisfaction perspective.	
\end{itemize}

We suggested to run standard statistical measurements on the data, in order to know, how representative it is for the whole visitor population. As a future question to investigate, using these results as well as the included R code (see listing \ref{lst:r-import}), a research could be made about creating more robust and reliable tests about internet site user satisfaction.

Ultimately, this document can not give profound advice with respect to the defined research questions, as only the pilot was carried out. Only general consultation can be given. In a first step, for each question as well as the the overall aggregation, a negative metric might be considered as a warning signal (the more negative the worse). A next step would be the investigation of variance. Third, a homogeneous negative response could point to a general issue while a group dependent negative rating might relate to possible intra-group improvements.


\section*{References}
\bibliographystyle{dinat}
\renewcommand\refname{\vskip -1cm}
\bibliography{ST3012_project_report_v19pette_v18tobwe}

\section{Appendices}

\subsection{Survey Guideline}
\textbf{Information about this questionnaire}\\
Thank you for cooperating by filling out this questionnaire. Please read this guideline carefully.
\\ \\
The aim of this research is improvement of the "personal search" sub-section of Dalarna University's internet site (https://www.du.se/). Your answers will be used and analyzed by the researchers of this project.
\\ \\
Your answers will be stored in a database and will be available for the researchers. Your personal details will be stored, however will be saved in a separate database. Only people who are taking part in this project have access to personal identification which will not be disclosed to anyone else. The rest of the data will be compiled and aggregated, so that it will not be possible to recognize individual answers or people from the report. The database will be deleted after 10 years.
\\ \\
\textbf{About the questions}\\
Think about the questions as carefully as possible and look for the answer which mostly describes your opinion. For every question there are 5 possible answers. They range from full agreement to the statement to full disagreement. Check the one which matches with your opinion. In the last question you can add any other notes about the "personal search" functionality.
\\ \\
Thank you.

\subsection{Survey Questions}
\begin{enumerate}
	\item Navigation to the "personal search" page is intuitive and fast?
	\item Loading time after entering the search term is quick? 
	\item Information you are looking was found immediately?
	\item It is intuitive to add/remove filters?
	\item If more than one filter is needed, is it clear what is the relation between two filters?
	\item Do you have any improvement suggestions?
\end{enumerate}

\subsection{Data Import into R}
\begin{lstlisting}[caption={Data Import into R and first analysis steps},label={lst:r-import},language=R]
# DS project
## Analysis of the survey data

## setup
#set random seed for reproducibility
set.seed(10)
setwd("/home/homaar/git/data-collection-project-19/")
#read google file
dt <- read_csv("Users' satisfaction about Dalarna University's Homepage.csv")
#convert to data frame
analysis <- data.frame(q1=1:nrow(dt))
# scale vector
scale <- c("I fully agree", "I agree", "I mainly agree", "I partly disagree", "I disagree")
#converion of question 1
analysis$q1 <- factor(dt[,5], levels=scale)
# apply correct scale (-2 -1 0 1 2)
analysis$q1numeric <- as.numeric(analysis$q1) - 3
\end{lstlisting}

\subsection{Google CSV file}
The Google CSV file can is attached as a separate document.

\subsection{DESAP checklist}
The DESAP checklist is attached as a separate document. The Excel version is used. Figure \ref{fig:desap} shows the graphical visualization of the answers.

\begin{figure}[h!]
    \centering
    \includegraphics[width=0.8\textwidth]{figures/desap-graph.PNG}
    \caption{DESAP checklist visualization}
    \label{fig:desap}
\end{figure}

\end{document}
